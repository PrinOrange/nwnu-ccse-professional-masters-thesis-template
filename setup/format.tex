%%%%%%%%%%%%%%%%%%%%%%%%%%%%%%%%%%%%%%%%%%%%%%%%%%%%%%%%%%%%%%%%%%%%%%
%  run mode: pdfLatex on WinEdt10.3
%  !Mode: "Tex:Utf-8"
%  Authors: Xiangdong Jia and Mangang Xie from UWC&IoT Lab
%  Copyright: 泛在无线通信与物联网实验室@计算机科学与工程学院@西北师范大学
%  Version: 2023 年 6 月 1 日
%%%%%%%%%%%%%%%%%%%%%%%%%%%%%%%%%%%%%%%%%%%%%%%%%%%%%%%%%%%%%%%%%%%%%%


%%%%%%%%%% Fonts Definition and Basics %%%%%%%%%%%%%%%%%
\renewcommand{\today}{二〇二三年五月}     %封面第一页的日期定义
\newcommand{\song}{\CJKfamily{song}}    % 宋体
%\newcommand{\FSGB}{\CJKfamily{FSGB}}
%\setCJKfamilyfont{FSGB}{仿宋_GB2312.ttf}
\newcommand{\FSGB}{\CJKfamily{FSGB}} %仿宋 GB2312
\newcommand{\fs}{\CJKfamily{fs}}        % 仿宋体
\newcommand{\kai}{\CJKfamily{kai}}      % 楷体
\newcommand{\hei}{\CJKfamily{hei}}      % 黑体
\newcommand{\li}{\CJKfamily{li}}        % 隶书
\newcommand{\yihao}{\fontsize{26pt}{26pt}\selectfont}       % 一号,1.倍行距
\newcommand{\xiaoyi}{\fontsize{24pt}{24pt}\selectfont}      % 小一,1.倍行距
\newcommand{\erhao}{\fontsize{22pt}{22pt}\selectfont}       % 二号,1.倍行距
\newcommand{\er}{\fontsize{20pt}{20pt}\selectfont}       % 20pt, 单倍行距
\newcommand{\xiaoer}{\fontsize{18pt}{18pt}\selectfont}      % 小二,单倍行距
\newcommand{\sanhao}{\fontsize{16pt}{16pt}\selectfont}      % 三号,1.倍行距
\newcommand{\xiaosan}{\fontsize{15pt}{15pt}\selectfont}     % 小三,1.倍行距
\newcommand{\sihao}{\fontsize{14pt}{14pt}\selectfont}       % 四号,1.0 倍行距
%\newcommand{\xiaosi}{\fontsize{12.5pt}{12.5pt}\selectfont}      % 小四,1.倍行距
\newcommand{\xiaosi}{\fontsize{12pt}{12pt}\selectfont}      % 小四,1.倍行距
\newcommand{\wuhao}{\fontsize{10.5pt}{10.5pt}\selectfont}   % 五号,单倍行距
\newcommand{\xiaowu}{\fontsize{9pt}{9pt}\selectfont}        % 小五,单倍行距

\setlength{\headheight}{24pt}%%%%%%%%%%%%页眉和上边距之间的距离%%%%%%%%%%%%%%%%%%%%%%%%%%%%%
%\setlength{\headsep}{20pt}
%\setlength{\footskip}{4pt}
\setlength{\textheight}{23cm}%%%%%%%%%%%%%%%%%%%%%%%%%%%%%%%%%%设置内容高度
%\newcommand{\abstractname}{摘要}
%\CJKcaption{gb_452}
\CJKtilde  % 重新定义了波浪符~的意义

\newcommand\prechaptername{第}
\newcommand\postchaptername{章}

% 调整罗列环境的布局
\setitemize{leftmargin=3em,itemsep=0em,partopsep=0em,parsep=0em,topsep=-0em}
\setenumerate{leftmargin=3em,itemsep=0em,partopsep=0em,parsep=0em,topsep=0em}


%避免宏包 hyperref 和 arydshln 不兼容带来的目录链接失效的问题。
\def\temp{\relax}
\let\temp\addcontentsline
\gdef\addcontentsline{\phantomsection\temp}

% 自定义项目列表标签及格式 \begin{publist} 列表项 \end{publist}
\newcounter{pubctr} %自定义新计数器
\newenvironment{publist}{%%%%%定义新环境
	\begin{list}{[\arabic{pubctr}]} %%标签格式
		{
		\usecounter{pubctr}
		\setlength{\leftmargin}{4em}     % 左边界 \leftmargin =\itemindent + \labelwidth + \labelsep
		\setlength{\itemindent}{0em}     % 标号缩进量
		      \setlength{\labelsep}{1em}       % 标号和列表项之间的距离,默认 0.5em
		      \setlength{\rightmargin}{0em}    % 右边界
		      \setlength{\topsep}{0ex}         % 列表到上下文的垂直距离
		      \setlength{\parsep}{0ex}         % 段落间距
		      \setlength{\itemsep}{0ex}        % 标签间距
		      \setlength{\listparindent}{0pt} % 段落缩进量
		      }}
		      {\end{list}}%%%%%

%%%%%%%%%%%%%%%%%%%%%%%%%%%%%%%%%%%%%%%%%%%%%%%%%%5默认字体大小
\makeatletter
\renewcommand\normalsize{
	\@setfontsize\normalsize{12pt}{12pt} % 小四对应 12pt
	\setlength\abovedisplayskip{4pt}
	\setlength\abovedisplayshortskip{4pt}
	\setlength\belowdisplayskip{\abovedisplayskip}
	\setlength\belowdisplayshortskip{\abovedisplayshortskip}
	\let\@listi\@listI}
%%%%%%%%%%%%%%%%%%%%%%%%-5555555555555                                 -------定义默认字体和行距---------%%%%%%%%%%%%%%%%%%%%%%%%%%%%%%%%%%%%%%%%%%%%%
%\renewcommand{\CJKglue}{\hskip 0.5pt \baselineskip}
\def\defaultfont{\renewcommand{\baselinestretch}{1.67}\normalsize\selectfont}

%\def\defaultfont{\renewcommand{\baselinestretch}{1.83}\normalsize\selectfont}
%\setlength{\baselineskip}{20pt}

% %%%%%%%%%%%%%%%%%%%%%%%%%%%%%%%%%%%%%%%%%%%%%%%%%%%%%%%%设置行距和段落间垂直距离

\setlength{\baselineskip}{18pt}
%\renewcommand{\CJKglue}{\hskip 0.5pt plus \baselineskip} %加大字间距,使每行 35 个字
%\renewcommand{\CJKglue}{\hskip 1cm \baselineskip} %加大字间距,使每行 35 个字
\renewcommand{\CJKglue}{\hskip 0.5pt plus \baselineskip} %加大字间距,使每行 35 个字
%\renewcommand{\CJKglue}{\hskip 2pt plus 0.08\baselineskip}
%\renewcommand{\CJKglue}{\hskip 1cm}

\makeatother

%%%%%%%%%%%%% Contents %%%%%%%%%%%%%%%%%
%\pagestyle{empty}
%\fancyhf{}
%\thispagestyle{empty}

\renewcommand{\contentsname}{目\quad 录}

%\thispagestyle{empty}
\setcounter{tocdepth}{2}%%%%%%%%设置目录层次 的深度
\titlecontents{chapter}[0em]{\xiaosi\hei}%
{\prechaptername~~\thecontentslabel~~\postchaptername~~~}{} %
{\titlerule*[5pt]{$\cdot$}\xiaosi\contentspage}
\titlecontents{section}[1em]{\xiaosi\song} %
{\thecontentslabel\quad }{} %
{\hspace{.25em}\titlerule*[5pt]{$\cdot$}\xiaosi\contentspage}
\titlecontents{subsection}[2em]{\xiaosi\song} %
{\thecontentslabel\quad }{} %
{\hspace{.25em}\titlerule*[5pt]{$\cdot$}\xiaosi\contentspage}
\renewcommand{\cftdotsep}{1.1}
\renewcommand{\listfigurename}{图片索引}
\setcounter{lofdepth}{1}
%\titlefigures{chapter}[1em]{\xiaosi\hei}%
%             {\prechaptername~~\thecontentslabel~~\postchaptername~~~}{} %
%            {\titlerule*[5pt]{$\cdot$}\xiaosi\contentspage}
\renewcommand{\listtablename}{表格索引}



%%%%%%%%%%%%%%%%%%%%%%%%%%%%%%%%%%%%%%%%%定义页眉和页脚%%%%%%%%%%%%%%%%%%%%%%%%%%%%%%%%%%%
\makeatletter

\def\@chapter[#1]#2{\ifnum \c@secnumdepth >\m@ne
		\if@mainmatter
			\refstepcounter{chapter}%
			\typeout{\@chapapp\space\thechapter.}%
			\addcontentsline{toc}{chapter}%
			{\protect\numberline{\thechapter}#1}%
			\def\leftmark{第\thechapter 章\quad #2}
		\else
			\addcontentsline{toc}{chapter}{#1}%
		\fi
	\else
		\addcontentsline{toc}{chapter}{#1}%
	\fi
	\chaptermark{#1}%
	\if@twocolumn
		\@topnewpage[\@makechapterhead{#2}]%
	\else
		\@makechapterhead{#2}%
		\@afterheading
	\fi}

\def\@schapter#1{\if@twocolumn
		\@topnewpage[\@makeschapterhead{#1}]%
	\else
		\@makeschapterhead{#1}%
		\def\leftmark{#1}
		\@afterheading
	\fi}
\makeatother
%%%%%%%%%%%%%%%%%%%%%%%%%%%%%%%%%%%%%%%%%%%%%%%%%%%%%%%%%%%%%%%%%%%%%%%%%%%%%%%%%%%%%%%%%%%%%%%%%%%

%%%%%%%%%% %%%%%%%%%%%%%%%%%-----定义章节和小节的地方---%%%%%%% %%%%%%%%%%%%%%%%%
\setcounter{secnumdepth}{4}
\setlength{\parindent}{2em}
\renewcommand{\chaptername}{\prechaptername\arabic{chapter}\postchaptername}
\titleformat{\chapter}{\centering\sanhao\hei}{\chaptername}{1em}{}%%%%%%%%%%%%{\centering}Large  定义章标题居左
\titlespacing{\chapter}{0pt}{-10pt}{18pt}
\titleformat{\section}{\sihao\hei}{\thesection}{1em}{}
\titlespacing{\section}{0pt}{20pt}{8pt}
\titleformat{\subsection}{\sihao\hei}{\thesubsection}{1em}{}
\titlespacing{\subsection}{0pt}{20pt}{8pt}
\titleformat{\subsubsection}{\xiaosi\hei}{\thesubsubsection}{0.5em}{}
\titlespacing{\subsubsection}{0pt}{6pt}{6pt}
%%%%%%%%%%%%%%%%%%%%%%%%%%%%%%%%%%%%%%%%%%%%%%%%%%%%%%%%%%%%%%%%%%%%%%%%%%%%%%%%%%%%%

%%%%%%%%%% %%%%%%%%%%%%%%%%%%%%%%%%%%%%%  图和表的编号 格式 %%%%%%%%%  %%%%%%%%%%%%%%%%%
\renewcommand{\tablename}{\bfseries 表} % 插表题头
\renewcommand{\figurename}{\bfseries 图} % 插图题头
%\renewcommand{\figurename}{{\bfseries 图~\arabic{chapter}-\arabic{figure}}} % 插图题头
%\renewcommand{\thefigure}{}

\renewcommand{\thefigure}{\arabic{chapter}-\arabic{figure}} % 使图编号为 7.1 的格式 %\protect{~}
\renewcommand{\thetable}{\arabic{chapter}-\arabic{table}}%使表编号为 7.1 的格式
\renewcommand{\theequation}{\arabic{chapter}-\arabic{equation}}%使公式编号为 7.1 的格式
\renewcommand{\thesubfigure}{(\alph{subfigure})}%使子图编号为 (a) 的格式
\renewcommand{\thesubtable}{(\alph{subtable})} %使子表编号为 (a) 的格式
\makeatletter
\renewcommand{\p@subfigure}{\thefigure~} %使子图引用为 7.1 a) 的格式,母图编号和子图编号之间用~ 加一个空格

%\captionsetup[figure]{labelformat=simple, labelsep=space, skip=88pt, labelfont={bf}}
\captionsetup[figure]{skip=8pt} % 很重要的一个语句

\makeatother


%% %%%%%%%%%%%%%%%%%%%%%%%%%%%%%%%%%定制浮动图形和表格标题样式%%%%%%%%%%%%%%%%%%
\makeatletter
\long\def\@makecaption#1#2{%
	\vskip\abovecaptionskip
	\sbox\@tempboxa{\centering\wuhao{#1~~#2} }%
	\ifdim \wd\@tempboxa >\hsize
		\centering\wuhao{#1~~#2} \par
	\else
		\global \@minipagefalse
		\hb@xt@\hsize{\hfil\box\@tempboxa\hfil}%
	\fi
	\vskip\belowcaptionskip}
\makeatother
\captiondelim{~~~~} %用来控制 longtable 表头分隔符

\newcommand{\figref}[1]{{图~\ref{#1}}}
\newcommand{\tabref}[1]{{表~\ref{#1}}}

%\newcommand{\figref}[1]{{图~\arabic{chapter}-\arabic{figure}\ref{#1}}}

%%%%%%%%%%%%%%%%%%%%%%%%%%%%%%%%%%%%%%%%%%%%%定理、定义、猜想%%%%%%%%%%%%%%%%%%%%%%%%%%%%%%%%%%%
\theorembodyfont{\song\rmfamily}
\theoremheaderfont{\hei\rmfamily}
\newtheorem{theorem}{\hei 定理~}[chapter]
\newtheorem{lemma}{\hei 引理~}[chapter]
\newtheorem{axiom}{\hei 公理~}[chapter]
\newtheorem{proposition}{命题~}[chapter]
\newtheorem{corollary}{推论~}[chapter]
\newtheorem{definition}{定义~}[chapter]
\newtheorem{conjecture}{猜想~}[chapter]
\newtheorem{example}{例~}[chapter]
\newtheorem{remark}{注~}[chapter]
\floatname{algorithm}{算法}%将英文的 algorithm 改为算法

\renewcommand{\algorithmicrequire}{\textbf{Input:}}
\renewcommand{\algorithmicensure}{\textbf{Output:}}
%\captionsetup[algorithm]{font=footnotesize}%将算法标题改为 10.5

\newcommand{\tabincell}[2]{\begin{tabular}{@{}#1@{}}#2\end{tabular}}%表格合并
\newenvironment{proof}{\noindent{\hei 证明:}}{\hfill $ \square $ \vskip 4mm}
\theoremsymbol{$\square$}


%%%%%%%%%%% JXD definition  %%%%%%%%%%%%%
%定义环境变量 TheoremJXD, 定理
\newcounter{Theo}[chapter]
% 每当使用\newcounter 命令定义一个新的计数器时,系统将会自动自动地定义一条新命令,
%\newcounter{\the 新计数器\arabic{新计数器}},此命令可用于显示该计数器当前值\the
\newenvironment{TheoremJXD}{\refstepcounter{Theo}\vspace{5pt}\indent \kaishu{\bfseries 定理\arabic{chapter}-\arabic{Theo}:}~}{\kaishu\vspace{-10pt}}
% 引用方式
\newcommand{\TheoremJXDref}[1]{{定理~\arabic{chapter}-\ref{#1}}}

%\newenvironment{myenv}{\noindent 文档前面的内容\par}{文档后面的内容}
%\renewcommand{\myenv}[2]{\arabic{chapter}-\arabic{myenv}:\kaishu}
%\renewcommand{\theTheo}{{{Theorem}~\arabic{Theo}}}

%定义环境变量 LemmaJXD, 引理
\newcounter{Lemm}[chapter]
% 每当使用\newcounter 命令定义一个新的计数器时,系统将会自动自动地定义一条新命令,
%\newcounter{\the 新计数器\arabic{新计数器}},此命令可用于显示该计数器当前值\the
\newenvironment{LemmaJXD}{\refstepcounter{Lemm}\vspace{5pt}\indent \kaishu{\bfseries 引理\arabic{chapter}-\arabic{Lemm}:}~}{\kaishu\vspace{-10pt}}
% 引用方式
\newcommand{\LemmaJXDref}[1]{{引理~\arabic{chapter}-\ref{#1}}}


%定义环境变量 CorollaryJXD, 推论
\newcounter{Corol}[chapter]
% 每当使用\newcounter 命令定义一个新的计数器时,系统将会自动自动地定义一条新命令,
%\newcounter{\the 新计数器\arabic{新计数器}},此命令可用于显示该计数器当前值\the
\newenvironment{CorollaryJXD}{\refstepcounter{Corol}\vspace{5pt}\indent \kaishu{\bfseries 推论\arabic{chapter}-\arabic{Corol}:}~}{\kaishu\vspace{-10pt}}
% 引用方式
\newcommand{\CorollaryJXDref}[1]{{推论~\arabic{chapter}-\ref{#1}}}


%定义环境变量 CorollaryJXD, 定义 Definition
\newcounter{Defi}[chapter]
% 每当使用\newcounter 命令定义一个新的计数器时,系统将会自动自动地定义一条新命令,
%\newcounter{\the 新计数器\arabic{新计数器}},此命令可用于显示该计数器当前值\the
\newenvironment{DefinitionJXD}{\refstepcounter{Defi}\vspace{5pt}\indent \kaishu{\bfseries 定义\arabic{chapter}-\arabic{Defi}:}~}{\kaishu\vspace{-10pt}}
% 引用方式
\newcommand{\DefinitionJXDref}[1]{{定义~\arabic{chapter}-\ref{#1}}}


%定义环境变量 PropositionJXD, 命题
\newcounter{Prop}[chapter]
% 每当使用\newcounter 命令定义一个新的计数器时,系统将会自动自动地定义一条新命令,
%\newcounter{\the 新计数器\arabic{新计数器}},此命令可用于显示该计数器当前值\the
\newenvironment{PropositionJXD}{\refstepcounter{Prop}\vspace{5pt}\indent \kaishu{\bfseries 命题\arabic{chapter}-\arabic{Prop}:}~}{\kaishu\vspace{-10pt}}
% 引用方式
\newcommand{\PropositionJXDref}[1]{{命题~\arabic{chapter}-\ref{#1}}}


%%%%%%%%%% Page: number, header and footer  页码%%%%%%%%%%%%%%%%%

%\frontmatter 或 \pagenumbering{roman}
%\mainmatter 或 \pagenumbering{arabic}
\makeatletter
\renewcommand\frontmatter{\clearpage
	\@mainmatterfalse
	\pagenumbering{Roman}} % 正文前罗马字体编号
\makeatother


%%%%%%%%%%%%%%%%%%%%%%%%%%%%%%%%定制附录 格式%%%%%%%%%%%%%%%%%


\newcounter{sectionAppen}
\renewcommand{\thesectionAppen}{{\arabic{sectionAppen}}}
\newcommand{\sectionAppenJXD}[1]{\par\noindent\refstepcounter{sectionAppen}{\sihao\textbf{%
			\thesectionAppen:~#1}}\par\setlength{\parindent}{2em}}

\newcounter{subsectionAppen}%{sectionApen}
\renewcommand{\thesubsectionAppen}{{\arabic{subsectionAppen}}}
\newcommand{\subsectionAppenJXD}[1]{\vspace{8pt}\par\noindent\refstepcounter{subsectionAppen}{\sihao\textbf{\thesectionAppen.
			\thesubsectionAppen:~#1}}\par\setlength{\parindent}{2em}}
\makeatletter
\@addtoreset{subsectionAppen}{sectionAppen}
\makeatother

\makeatletter

\newcommand{\AppendixJXDtitle}[2]{\clearpage{\quad\vspace{-1.2em}}\par
	\sanhao\hei\centering#1\vspace{1em}\par\centering#2\par\raggedright}

\renewcommand\appendix{\par\raggedright
	\setcounter{sectionAppen}{0}
	\setcounter{subsectionAppen}{0}
	\gdef\thesectionAppen{附录 \Roman{chapter}-\Alph{sectionAppen}~\vspace{8pt}}\par
	\renewcommand{\theequation}{\Roman{chapter}-\Alph{sectionAppen}-\arabic{equation}}
	\setcounter{equation}{0}\par
	\@addtoreset{equation}{sectionAppen}\justifying
}
\makeatother

%\renewcommand\appendix{\par\raggedright
%	\setcounter{section}{0}
%	\setcounter{subsection}{0}
%	\gdef\thesection{附录 \Roman{chapter}-\Alph{section}}\par
%	\renewcommand{\theequation}{\Roman{chapter}-\Alph{section}-\arabic{equation}}
%	\setcounter{equation}{0}
%}

%\renewcommand\appendix{\par\raggedright
%	\setcounter{section}{0}
%	\setcounter{subsection}{0}
%	\gdef\thesection{附录 \Roman{chapter}-\Alph{section}}\par
%	\renewcommand{\theequation}{\Roman{chapter}-\Alph{section}-\arabic{equation}}
%	\setcounter{equation}{0}
%}
\newcommand{\AppendixJXD}[2]{{\AppendixJXDtitle{#1}{#2}}\appendix{ }}
%\centering\appendix
%%


\makeatletter
\newenvironment{AppendixM}{\addcontentsline{toc}{section}{\song\quad 本章附录} \par
	\indent{}}{\hfill $\square $ \vskip 4mm\par
	\renewcommand{\theequation}{\arabic{chapter}-\arabic{equation}\par}
}

\makeatother
%%
%%%%%%%%%% References %%%%%%%%%%%%%%%%%
\renewcommand{\bibname}{参考文献}
% 重定义参考文献样式,来自 thu
\makeatletter
\renewenvironment{thebibliography}[1]{%
	\chapter*{\bibname}%
	\wuhao
	\list{\@biblabel{\@arabic\c@enumiv}}%
	{\renewcommand{\makelabel}[1]{##1\hfill}
		\setlength{\baselineskip}{16pt}
		\settowidth\labelwidth{0.5cm}
		\setlength{\labelsep}{0pt}
		\setlength{\itemindent}{0pt}
		\setlength{\leftmargin}{\labelwidth+\labelsep}
		\addtolength{\itemsep}{-0.7em}
		\usecounter{enumiv}%
		\let\p@enumiv\@empty
		\renewcommand\theenumiv{\@arabic\c@enumiv}}%
	\sloppy\frenchspacing
	\clubpenalty4000%
	\@clubpenalty \clubpenalty
	\widowpenalty4000%
	\interlinepenalty4000%
	\sfcode`\.\@m}
{\def\@noitemerr
	{\@latex@warning{Empty `thebibliography' environment}}%
	\endlist\frenchspacing}
\makeatother

\addtolength{\bibsep}{5pt} % 增加参考文献间的垂直间距
\setlength{\bibhang}{2em} %每个条目自第二行起缩进的距离

% 参考文献引用作为上标出现
\newcommand{\scite}[1]{\scalebox{1.1}[1.1]{\textsuperscript{\cite{#1}}}}
\newcommand{\mycite}[1]{\scalebox{1.3}[1.3]{\raisebox{-0.65ex}{\cite{#1}}}}
%\newcommand{\citenormal}[1]{\cite{#1}}
%\makeatletter
%   \def\@cite#1#2{\textsuperscript{[{#1\if@tempswa , #2\fi}]}}
%\makeatother

%% 引用格式
\bibpunct{[}{]}{,}{s}{}{,}

%%%%%%%%%%%%%%%%%%%%%%%%%% -----------Cover----------------- %%%%%%%%%%%%%%%%%
% %%%%%%%%%%%%%%%%%%%%%%%%---封面、摘要、版权、致谢格式定义--%%%%%%%%%%%%%%%%%%%
\makeatletter

%\def\dtitle#1{\def\@dtitle{#1}}\def\@dtitle{}
\def\ctitle#1{\def\@ctitle{#1}}\def\@ctitle{}
\def\etitle#1{\def\@etitle{#1}}\def\@etitle{}
\def\caffil#1{\def\@caffil{#1}}\def\@caffil{}
\def\cmacrosubject#1{\def\@cmacrosubject{#1}}\def\@cmacrosubject{}
\def\cmacrosubjecttitle#1{\def\@cmacrosubjecttitle{#1}}\def\@cmacrosubjecttitle{}
\def\csubject#1{\def\@csubject{#1}}\def\@csubject{}
\def\csubjecttitle#1{\def\@csubjecttitle{#1}}\def\@csubjecttitle{}
\def\cmajor#1{\def\@cmajor{#1}}\def\@cmajor{}
\def\cauthor#1{\def\@cauthor{#1}}\def\@cauthor{}
\def\cauthortitle#1{\def\@cauthortitle{#1}}\def\@cauthortitle{}
\def\csupervisor#1{\def\@csupervisor{#1}}\def\@csupervisor{}
\def\csupervisortitle#1{\def\@csupervisortitle{#1}}\def\@csupervisortitle{}
\def\cdate#1{\def\@cdate{#1}}\def\@cdate{}%%%%%%%%今天的日期
\def\untitle#1{\def\@untitle{#1}}\def\@untitle{}
\def\declaretitle#1{\def\@declaretitle{#1}}\def\@declaretitle{}
\def\authorinformationtitle#1{\def\@authorinformationtitle{#1}}\def\@authorinformationtitle{}
%%%%%%%%%%%%%%%%%%%%%%%--封面第一页用到的定义---%%%%%%%%%%%%%%%%%%%%%%%%%%%%%%%%%%%%
\def\cheadinga#1{\def\@cheadinga{#1}}\def\@cheadinga{}
\def\cheadingb#1{\def\@cheadingb{#1}}\def\@cheadingb{}
%%%%%%%%%%%%%%%%%%%%%%%%%%%%%%%%%%%%%%%%%%%%%%%%%%%%%%%%%%%%%%%%%%%%%%%%%%%%%%%%%%%%%
%%%%%%%%%%%%%%%%%%%%------西北师范大学学位论文版权使用授权书 - 用到的定义-------------%%%%%%%%%%%%%%%%
\def\banquanshiyongshouquan#1{\def\@banquanshiyongshouquan{#1}}\def\@banquanshiyongshouquan{}
\def\banquanshiyongshouquanzhengwena#1{\def\@banquanshiyongshouquanzhengwena{#1}}\def\@banquanshiyongshouquanzhengwena{}
\def\banquanshiyongshouquanzhengwenb#1{\def\@banquanshiyongshouquanzhengwenb{#1}}\def\@banquanshiyongshouquanzhengwenb{}
\def\gongjuren#1{\def\@gongjuren{#1}}\def\@gongruren{}
%%%%%%%%%%%%%%%%%%%%%%%%%%%%%%%%%%%%%%%%%%%%%%%%%%%%%%%%%%%%%%%%%%%%%%%%%%%%%%%%%%%%%%%%%%%%%%%%%%%%%%%

\def\declarecontent#1{\def\@declarecontent{#1}}\def\@declarecontent{}
\def\authorizationtitle#1{\def\@authorizationtitle{#1}}\def\@authorizationtitle{}
\def\authorizationcontent#1{\def\@authorizationcontent{#1}}\def\@authorizationconent{}
\def\authorizationcontentone#1{\def\@authorizationcontentone{#1}}\def\@authorizationconentone{}
\def\authorizationcontenttwo#1{\def\@authorizationcontenttwo{#1}}\def\@authorizationconenttwo{}
\def\authorizationcontentthree#1{\def\@authorizationcontentthree{#1}}\def\@authorizationconentthree{}
\def\authorizationadd#1{\def\@authorizationadd{#1}}\def\@authorizationadd{}
\def\ownersigncap#1{\def\@ownersigncap{#1}}\def\@ownersigncap{}

\def\authorsigncap#1{\def\@authorsigncap{#1}}\def\@authorsigncap{}
\def\supervisorsigncap#1{\def\@supervisorsigncap{#1}}\def\@supervisorsigncap{}
\def\signdatecap#1{\def\@signdatecap{#1}}\def\@signdatecap{}
\long\def\cabstract#1{\long\def\@cabstract{#1}}\long\def\@cabstract{}
\long\def\eabstract#1{\long\def\@eabstract{#1}}\long\def\@eabstract{}
\def\ckeywords#1{\def\@ckeywords{#1}}\def\@ckeywords{}
\def\ekeywords#1{\def\@ekeywords{#1}}\def\@ekeywords{}
\def\cheading#1{\def\@cheading{#1}}\def\@cheading{}
\def\cnumber#1{\def\@cnumber{#1}}\def\@cnumber{}
\def\csecret#1{\def\@csecret{#1}}\def\@csecret{}
\def\chnunumer#1{\def\@chnunumer{#1}}\def\@chnunumer{}
\def\cclassnumber#1{\def\@cclassnumber{#1}}\def\@cclassnumber{}
\def\chnuname#1{\def\@chnuname{#1}}\def\@chnuname{}
\def\cchair#1{\def\@cchair{#1}}\def\@cchair{}
\def\ddate#1{\def\@ddate{#1}}\def\@ddate{}
%英文内封
\def\ename#1{\def\@ename{#1}}\def\@ename{}
\def\cbe#1{\def\@cbe{#1}}\def\@cbe{}
\def\cms#1{\def\@cms{#1}}\def\@cms{}
\def\cdegree#1{\def\@cdegree{#1}}\def\@cdegree{}
\def\cclass#1{\def\@cclass{#1}}\def\@cclass{}
\def\emajor#1{\def\@emajor{#1}}\def\@emajor{}
\def\ehnu#1{\def\@ehnu{#1}}\def\@ehnu{}
\def\esupervisor#1{\def\@esupervisor{#1}}\def\@esupervisor{}
\def\echair#1{\def\@esupervisor{#1}}\def\@echair{}%xinzeng
\def\edate#1{\def\@edate{#1}}\def\@edate{}
\def\elevel#1{\def\@elevel{#1}}\def\@elevel{}


%%%%%%%%%%%%%%%%%%%%%%%%%%%%%--------第一页封面距离定义---------%%%%%%%%%%%%%%%%%%%%%%%
\newlength{\@title@width}
\newlength{\@titlea@width}
\newlength{\@titleb@width}
\newlength{\@titlec@width}
%%%%%%%%%%%%%%%%%%%%%%%%%%%%%%%%%%%%%%%%%%%%%%%%%%%%%%%%%%%%%%%%%%%%%%%%%%%%%%%%%%%%%
%%%%%%%%%%%%%%%%%%%%%%%%%%%%%%%%%%%%%%%%%%%开始的地方%%%%%%%%%%%%%%%%%%%%%%%%%%%%%%%%%%%%%%%%%%%%%%
\def\@put@covertitle#1{\makebox[\@title@width][s]{#1}}
% 定义封面
\def\makecover{
	%\cleardoublepage%
	\phantomsection
	\pdfbookmark[-1]{\@ctitle}{ctitle}

	\begin{titlepage}
		\begin{center}

			\setlength{\@title@width}{1.6cm}%----------------------对封面分类号后面横线的长度进行定义
			{
				\begin{spacing}{2.1}
					\begin{tabular}{lcclc}

						\sihao\fs{~~分~类~号} & \underline{\makebox[\@title@width][c]{\@chnunumer}}    & \qquad \qquad \qquad \qquad \qquad \qquad \qquad & \sihao\fs{密~级} & \underline{\makebox[\@title@width][c]{\@cnumber}}          \\
						\sihao\fs{~~U~~D~~C}  & \underline{\makebox[\@title@width][c]{\@cclassnumber}} & \qquad \qquad \qquad \qquad \qquad \qquad \qquad & \sihao\fs{编~号} & \underline{\makebox[\@title@width][c]{\sihao\fs\@csecret}} \\
					\end{tabular}
				\end{spacing}
			}
			%%%%%%%%%%%%%%%%%%%%%%西北师范大学 logo 图片%%%%%%%%%%%%%%%%%%%%%%%%
			\begin{figure}[h]
				\centering
				\vspace*{0.6cm}
				\includegraphics[width=0.4\textwidth]{figures/nwnulog}
			\end{figure}
			%%%%%%%%%%%%%%%%%%%%%%%%%%%硕士学位论文%%%%%%%%%%%%%%%%%%%%%%%%%
			\vspace*{0.3cm}
			{\song\erhao\@cheadinga}\\
			\vspace{0.4\baselineskip}

			{\song\sanhao\@cheadingb}

			\vspace*{1cm}

			%%%%%%%%%%%%%%%%%%%%%%%%%%%%%%论文标题%%%%%%%%%%%%%%%%%%%%%%%%%%%%%%%%%%%%%
			\begin{center}
				\begin{spacing}{1.5}
					\kai \yihao \@ctitle
				\end{spacing}
			\end{center}

			\vspace{\baselineskip}
			\setlength{\@title@width}{6.0cm}
			\setlength{\@titlea@width}{5.0cm}
			\setlength{\@titleb@width}{5.68cm}
			\setlength{\@titlec@width}{5.9cm}
			{
				%%%%%%%%%%%%%%%%%%%%毕业生信息等%%%%%%%%%%%%%%%%%%%%%%%%%%%%%%%
				\begin{spacing}{2.0}%1.67
					%\vspace*{0.5 \baselineskip}
					\sihao\song {研~~~~究~~~~生~~~~姓~~~~名:} \sihao\song\underline{\makebox[\@title@width][c]{\@cauthor}} \\
					\sihao\song{指导教师姓名、职称:} \sihao\song\underline{\makebox[\@title@width][c]{\@csupervisor}} \\
					\sihao\song{实践指导教师姓名、职称:} \sihao\song\underline{\makebox[\@titlea@width][l]{\quad~\@cchair}} \\
					\sihao\song {专~~~业~~~学~~~位~~~类~~~别:} \sihao\song\underline{\makebox[\@titleb@width][c]{\@csubjecttitle}} \\
					\sihao\song {专~~~业~~~学~~~位~~~领~~~域:} \sihao\song\underline{\makebox[\@titleb@width][c]{\@caffil}} \\
					\sihao\song{专~~~~~~~项~~~~~~~计~~~~~~~划:} \sihao\song\underline{\makebox[\@titlec@width][l]{\qquad\@cmajor}}\\
					\vspace{2.0\baselineskip}
					%\vspace*{2.5cm}

					\sanhao\hei{\makebox[\@title@width][l]{~~~~~~~~~~\today}} \\%%%封面第一页的日期

					%\end{tabular}    \qquad
				\end{spacing}

			}
		\end{center}

		%%%%%%%%%%%%%%%%%%%%%%%%%%%%%%%%%%%%%%%%%%%%%%%%%%%%%%%%%%%%%%%%%%%%%%%%%%%%%
		%\clearpage
		%\thispagestyle{empty} %去掉页眉页脚
		%
		%\noindent
		%\makebox[2.59cm][s]{}{\begin{tabular}{ll}
		%\xiaosi\hei 学校代号:\xiaosi\song~~\@chnunumer \\
		%\xiaosi\hei 学\qquad~号:\xiaosi\song~~\@cnumber\\
		%\xiaosi\hei 密\qquad~级:\xiaosi\song~~\@csecret\\
		%\end{tabular}
		%}
		%
		%%
		%\vspace{5\baselineskip}
		%
		%\noindent
		%\makebox[2.59cm][s]{}{
		%\xiaoer\song \@chnuname \@cheading
		%}
		%\\
		%\vspace{4\baselineskip}
		%
		%\begin{spacing}{2}
		%\hangafter=1\hangindent=2.7cm   %换行后自动缩进
		%{\noindent
		%\makebox[2.59cm][s]{} {\hei\erhao\@ctitle}}
		%\end{spacing}
		%\vspace{4\baselineskip}
		%
		%\setlength{\@title@width}{6.5cm}
		%  {
		%  \begin{spacing}{2}
		%  \xiaosi
		%  \noindent
		%  \makebox[2.59cm][s]{}{
		%    \begin{tabular}{lc}
		%   \underline{\xiaosi\hei 学位申请人姓名:\song\makebox[\@title@width][l]{\qquad\qquad\@cauthor}} \\
		%   \underline{\xiaosi\hei 导师姓名及职称:\song\makebox[\@title@width][l]{\qquad\qquad\@csupervisor}} \\
		%   \underline{\xiaosi\hei 培~~~~养~~~~~单~~~~位:\song\makebox[\@title@width][l]{\qquad\qquad\@caffil}} \\
		%   \underline{\xiaosi\hei 专~~~~业~~~~~名~~~~称:\song\makebox[\@title@width][l]{\qquad\qquad\@csubject}} \\
		%   \underline{\xiaosi\hei 论~文~提~交~日~期:\song\makebox[\@title@width][l]{\qquad\qquad\@cdate}} \\
		%   \underline{\xiaosi\hei 论~文~答~辩~日~期:\song\makebox[\@title@width][l]{\qquad\qquad\@ddate}}\\
		%   \underline{\xiaosi\hei 答辩委员会主席:\song\makebox[\@title@width][l]{\qquad\qquad\@cchair}} \\
		%  \end{tabular}
		%   }
		%   \end{spacing}
		%    }

		%%%%%%%%%%%%%%%%%%%%%%%%%%%%英文题目%%%%%%%%%%%%%%%%%%%%%%%%%%
		\clearpage
		\thispagestyle{empty} %去掉页眉页脚

		\begin{center}
			\qquad\\
			\vspace*{2.1cm}
			\begin{spacing}{1.5}
				\er\@etitle

			\end{spacing}

			\begin{spacing}{2.0}
				\vspace*{1.85\baselineskip}
				\sihao
				A Thesis Submitted to\\
				Northwest Normal University\\
				in partial fulfillment of the requirement\\
				for the degree of\\
				Master of Electronic Information / Education\\
				by\\
				\@ename \\
				Supervisor: (Associate) Professor \@esupervisor\\
				Supervisor for Practice Guiding: Senior Engineer Wang Weimin

				\vspace*{3\baselineskip}
				\sanhao\@edate


			\end{spacing}
		\end{center}


	\end{titlepage}

	%%%%%%%%%%%%%%%%%%%%%%%%%%-------西北师范大学学位论文原创性声明----%%%%%%%%%%%%%%%%%%%%%%%%%%%%%%%%%
	%\hspace*{2em}
	\clearpage
	\thispagestyle{empty}
	\begin{center}\hei\xiaoer{\@gongjuren}\end{center}\par
	\vspace{2.8\baselineskip}

	\begin{center}\hei\xiaoer{\@authorizationtitle}\end{center}\par
	\vspace{1.8\baselineskip}

	{
		\linespread{2.0}
		\song\sihao{\@authorizationcontent}\par

		% \song\defaultfont{\@authorizationcontentone}\par
		%  \song\defaultfont{\@authorizationcontenttwo}\par
		%            \song\defaultfont{\@authorizationcontentthree}\par
		%    \begin{tabular}{ll}    @authorizationconentone
		%     \song\defaultfont\@authorizationadd\par&\\
		%    &1、保密\song\xiaoer{$\Box$}\song\xiaosi,在\underline{\qquad}年解密后适用于本授权书\\
		%    &2、不保密\song\xiaoer{$\Box$}。\\
		%    &(请在以上相应方框内打"$\surd$") \\
		%    \end{tabular}
	}
	\vspace{2.55\baselineskip}
	\begin{flushright}
		{
			\song\sihao
			\@ownersigncap \makebox[3.5cm][s]{}\par  %\@signdatecap \makebox[1.5cm][s]{} 年 \makebox[1cm][s]{} 月 \makebox[1cm][s]{} 日 \\
			%\indent
			\vspace{0.6\baselineskip}
			\@supervisorsigncap \makebox[3.5cm][s]{}\par
			\vspace{0.6\baselineskip}
			\@signdatecap\makebox[1.5cm][s]{}
			%\makebox[0.7cm][s]{} 年 \makebox[0.7cm][s]{} 月 \makebox[0.7cm][s]{} 日
			%\vspace{0.6\baselineskip}

		}
	\end{flushright}


	%%%%%%%%%%%%%%%%%%%%%%%%%%-----西北师范大学学位论文版权使用授权书---%%%%%%%%%%%%%%%%%%%%%%%%%%%%%%%%%

	\clearpage
	\thispagestyle{empty}
	\begin{center}\hei\xiaoer{\@gongjuren}\end{center}\par
	\vspace{2.5\baselineskip}

	\begin{center}\hei\xiaoer{\@banquanshiyongshouquan}\end{center}\par
	\vspace{2\baselineskip}
	{
		%\setlength{\baselineskip}{18pt}
		\linespread{2.0}
		\song\sihao{\@banquanshiyongshouquanzhengwena}\par
		\song\sihao{\@banquanshiyongshouquanzhengwenb}\par

		% \song\defaultfont{\@authorizationcontentone}\par
		%  \song\defaultfont{\@authorizationcontenttwo}\par
		%            \song\defaultfont{\@authorizationcontentthree}\par
		%    \begin{tabular}{ll}    @authorizationconentone
		%     \song\defaultfont\@authorizationadd\par&\\
		%    &1、保密\song\xiaoer{$\Box$}\song\xiaosi,在\underline{\qquad}年解密后适用于本授权书\\
		%    &2、不保密\song\xiaoer{$\Box$}。\\
		%    &(请在以上相应方框内打"$\surd$") \\
		%    \end{tabular}
	}
	\vspace{2\baselineskip}
	\begin{flushright}
		{
			\song\sihao
			\@ownersigncap \makebox[3.5cm][s]{}\par  %\@signdatecap \makebox[1.5cm][s]{} 年 \makebox[1cm][s]{} 月 \makebox[1cm][s]{} 日 \\
			%\indent
			\vspace{0.6\baselineskip}
			% \@supervisorsigncap \makebox[3.5cm][s]{}\par
			%      \vspace{0.6\baselineskip}
			\@signdatecap \makebox[1.4cm][s]{}
			%\vspace{1\baselineskip}
			%\makebox[0.8cm][s]{} 年 \makebox[0.8cm][s]{} 月 \makebox[0.8cm][s]{} 日
		}
	\end{flushright}
	%}
	%%%%%%%%%%%%%%%%%%%%%%%%%%%%%%%%%%西北师范大学研究生学位论文作者信息%%%%%%%%%%%%%%%%%%%%%%%%%%%%%%%%%%%%%%%%%%%%%%%%%
	%\clearpage
	%\thispagestyle{empty} %去掉页眉页脚
	%\fancyfoot[C]{\song\xiaowu ~\thepage~}
	%{
	%    \vspace{0.6\baselineskip}
	%    \begin{center}\hei\xiaoer{\@authorinformationtitle}\end{center}\par
	%    %表格
	%
	%       \renewcommand\arraystretch{2.5}
	%\begin{table}[h]
	%\centering
	%\song\xiaosi
	%%\resizebox{\textwidth}{10cm}{%表格宽度减小
	%\setlength{\tabcolsep}{7mm}{%表格宽度增加
	%\begin{tabular}{|c|c|c|c|c|}
	%  %\song\xiaosi
	%  \hline
	%  % after \\: \hline or \cline{col1-col2} \cline{col3-col4} ...
	%  %跨列(将两列合并为一列)第一个参数指明跨几列  第二个参数指明内容居中并在左右两边画上直线 最后一个参数是表格内容 multicolumn{2}{|c|}{1}
	%  \song\xiaosi{论文题目}& \multicolumn{4}{c|}{基于小区分割的 5G 异构网络资源配置和干扰协调方案} \\
	%  \hline
	%  姓~~~~名\song\xiaosi &\multicolumn{2}{c|}{纪珊珊}& \song\xiaosi{学~~~~号}&\song\xiaosi{2016211338}\\
	%  \hline
	%  专业名称\song\xiaosi &\multicolumn{2}{c|}{软件工程}&\song\xiaosi{答辩日期} &  \\
	%  \hline
	%  联系电话\song\xiaosi & 18252086081 &${\text{E\_mail}}$ & \multicolumn{2}{c|}{1347920139$\text{@}$qq.com} \\
	%  \hline
	%  \multicolumn{5}{|c|}{通信地址(邮编):兰州市安宁区安宁东路 967 号西北师范大学,730070}\\
	%    \hline
	%  \multicolumn{5}{|l|}{备注:} \\
	%   \hline
	%\end{tabular}
	%}
	%\end{table}
	%
	%}


	%%%%%%%%%%%%%%%%%%%%%%%%%%%%%%%%%%%%%%%%%%%%%%%%%%%%%%%%%%%%%%%%%%%%%%%%%%%%%%%%%%%%
	%%%%%%%%%%%%%%%%%%%%%%%%%%%%%%%   Abstract and Keywords  %%%%%%%%%%%%%%%%%%%%%%%%%%%%%%%%%%
	%\clearpage
	%
	%\pagestyle{fancy}
	%  \fancyhf{}
	%\fancyhead[CO]{\song\xiaowu \@cheading}
	%\fancyhead[CE]{\song\xiaowu \@ctitle}
	%\fancyfoot[C]{\song\xiaowu ~\thepage~}
	%%\fancyhead[CO]{\xiaowu\song\leftmark} %奇数页,章标题
	%\fancyhead[LE]{\small\songti\rightmark}
	%\fancyhead[RO]{\small\songti\leftmark}



	%\makeatletter %双线页眉
	%\def\headrule{{\if@fancyplain\let\headrulewidth\plainheadrulewidth\fi%
	%\hrule\@height 1.0pt \@width\headwidth\vskip1pt %上面线为 1pt 粗
	%\hrule\@height 0.5pt\@width\headwidth  %下面 0.5pt 粗
	%\vskip-2\headrulewidth\vskip-1pt}      %两条线的距离 1pt
	%\vspace{7mm}
	%}     %双线与下面正文之间的垂直间距
	%
	%\fancypagestyle{plain}{% 设置开章页页眉页脚风格
	%   \fancyhf{}%
	%\fancyhead[CO]{\song\xiaowu \@cheading}
	%\fancyhead[CE]{\song\xiaowu \@ctitle}
	%\fancyfoot[C]{\song\xiaowu ~\thepage~} %首页页脚格式
	%}


	%\addcontentsline{toc}{chapter}{摘~~~~~~~~要}
	% \chapter*{\centering\xiaoer\ 摘\qquad 要}
	% \song\defaultfont
	% \@cabstract
	% \vspace{\baselineskip}

	% \hangafter=1\hangindent=52.3pt\noindent   %如果取消该行注释,关键词换行时将会自动缩进
	% \noindent
	% {\hei\xiaosi 关键词: \@ckeywords}
	%
	%%%%%%%%%%%%%%%%%%%%   English Abstract  %%%%%%%%%%%%%%%%%%%%%%%%%%%%%%
	%\clearpage
	%
	%\addcontentsline{toc}{chapter}{Abstract}
	% \chapter*{\centering\xiaoer \bf{Abstract}}
	% %\vspace{\baselineskip}
	% \@eabstract
	% \vspace{\baselineskip}

	% \hangafter=1\hangindent=60pt\noindent  %如果取消该行注释,KEY WORDS 换行时将会自动缩进
	% \noindent
	% {\xiaosi\textbf{Key Words:} \@ekeywords}
}
%\clearpage
%\makeatother
%\thispagestyle{empty}
%---------------------------------------------------------------------------%
%-
%\RequirePackage{fancyhdr}
%\setlength{\headheight}{15pt}
%\fancypagestyle{plain}{%为了章首页
%\fancyhf{}
%\fancyhead[OC]{\zihao{5}\songti\leftmark} %奇数页,章标题
%\fancyhead[OR]{\zihao{-5}\thepage}
%\fancyhead[EC]{\zihao{5}\songti\@ctitle} %论文题目
%\fancyhead[EL]{\zihao{-5}\thepage}
%\renewcommand\headrulewidth{0.75pt}
%\renewcommand{\footrulewidth}{0pt}}
%
%\fancypagestyle{myheadings}{%为了摘要
%\fancyhf{}
%\fancyhead[C]{\zihao{5}\songti\XDU@abstractname}
%\renewcommand{\headrulewidth}{0.75pt}
%\renewcommand{\footrulewidth}{0pt}}
%\fancypagestyle{headings}{%为了 abstract
%\fancyhf{}
%\fancyhead[C]{\zihao{5}\songti\XDU@enabstractname}
%\renewcommand{\headrulewidth}{0.75pt}
%\renewcommand{\footrulewidth}{0pt}}
%
%\def\ps@XDU@mulu{%
%  \let\@oddhead\@empty%
%  \let\@evenhead\@empty%
%  \let\@oddfoot\@empty%
%  \let\@evenfoot\@empty}
%
%\fancypagestyle{XDU@mulu}{%为了目录
%\fancyhf{}
%\fancyhead[OC]{\zihao{5}\songti\leftmark}
%\fancyhead[OR]{\zihao{-5}\thepage}
%\fancyhead[EC]{\zihao{5}\songti\leftmark}
%\fancyhead[EL]{\zihao{-5}\thepage}
%\renewcommand{\headrulewidth}{0.75pt}
%\renewcommand{\footrulewidth}{0pt}}                                                                   -%


