\addcontentsline{toc}{chapter}{结\quad 论}
\chapter*{结\quad 论}

学位论文的结论作为论文正文的最后一章单独排写,但不加章标题序号。
结论应是作者在学位论文研究过程中所取得的创新性成果的概要总结,不能与摘要混为一谈。学位论文结论应包括论文的主要结果和创新点、研究展望两部分,在结论中应概括论文的核心观点,明确、客观地指出本研究内容的创新性成果(含新见解、新观点、方法创新、技术创新、理论创新),并指出今后进一步在本研究方向进行研究工作的展望与设想。对所取得的创新性成果应注意从定性和定量两方面给出科学、准确的评价,分(1)、(2)、(3)…条列出,宜用“提出了”、“建立了”等词叙述。
结论是对论文主要研究结果、论点的提炼与概括,应准确、简明,完整,有条理,使人看后就能全面了解论文的意义、目的和工作内容。主要阐述自己的创造性工作及所取得的研究成果在本学术领域中的地位、作用和意义。同时,要严格区分自己取得的成果与导师及他人的科研工作成果。
在评价自己的研究工作成果时,要实事求是,除非有足够的证据表明自己的研究是“首次”的,“领先”的,“填补空白”的,否则应避免使用这些或类似词语。

