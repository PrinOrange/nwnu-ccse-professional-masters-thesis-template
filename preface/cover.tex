%%%%%%%%%%%%%%%%%%%%%%%%%%%%%%%%%%%%%%%%%%%%%%%%%%%%%%%%%%%%%%%%%%%%%%
%  run mode: pdfLatex on WinEdt10.3
%  !Mode: "Tex:Utf-8"
%  Authors: Xiangdong Jia and Mangang Xie from UWC&IoT Lab
%  Copyright: 泛在无线通信与物联网实验室@计算机科学与工程学院@西北师范大学
%  Version: 2023 年 6 月 1 日
%%%%%%%%%%%%%%%%%%%%%%%%%%%%%%%%%%%%%%%%%%%%%%%%%%%%%%%%%%%%%%%%%%%%%%
\chnunumer{}%分类号
\cclassnumber{}%UDC
\cnumber{}%密级
\csecret{10736}%编号


\cmajor{\centering }
\cheadinga{硕~士~学~位~论~文}
\cheadingb{(专~业~学~位)}
\cheading{西北师范大学硕士学位论文}
\ctitle{面向 6G 物联网系统的多源优化调度与信息及时性研究}  %封面中文标题,自己可手动断行,副标题可删掉
\etitle{\textbf{Multi-source Optimal Scheduling and Information Timeliness for 6G Internet of Things System}}        %封面英文标题


\cauthor{李伟明}                                  %研究生姓名
\csupervisor{张伟明~~教授}                        %指导教师姓名、职称:教授/副教授/正高级工程师/高级工程师/研究员/副研究员
\cchair{王伟明~~高级工程师}                        %实践指导教师姓名、职称:教授/副教授/正高级工程师/高级工程师/研究员/副研究员
\csubjecttitle{电子信息}                          %专业学位类别:电子信息/职业技术教育
\caffil{计算机技术}                               %专业学位领域:计算机技术/软件工程/人工智能


\ename{Li Weiming}                               %研究生姓名英文
\esupervisor{Zhang Weiming}                      %指导教师姓名英文
\elevel{Professor}                               %指导教师职称英文
%\echair{Zhang Wuji}                             %实践导师英文
\edate{May,~2023}                                %时间英文


\authorizationtitle{西北师范大学学位论文原创性声明}
%\vspace{4em}
\gongjuren{~~~~~~~~}
\authorizationcontent{
	本人郑重声明:所呈交的学位论文是本人在导师的指导下独立进行研究工作所取得的成果。除文中已经注明引用的内容外,本论文不含任何其他个人或集体已经发表或撰写过的作品成果。对本文的研究做出重要贡献的个人和集体,均已在文中以明确方式标明。因本学位论文引起的法律后果完全由本人承担。
}

%%%%%%%%%%%%%%%%%%%%%%%%%%-----西北师范大学学位论文版权使用授权书---%%%%%%%%%%%%%%%%%%%%%%%%%%%%%%%%%
\banquanshiyongshouquan{西北师范大学学位论文版权使用授权书}
\banquanshiyongshouquanzhengwena{本学位论文作者完全了解西北师范大学有关保留、使用学位论文的规定,有权保留并向国家有关部门或机构送交论文的纸质版和电子版,允许论文被查阅和借阅。本人授权西北师范大学可以将学位论文的全部或部分内容编入有关数据库进行检索,可以采用影印、缩印或扫描等复制手段保存、汇编学位论文,可以公开学位论文的全部或部分内容。
}
\banquanshiyongshouquanzhengwenb{(保密的学位论文在解密后适用本授权书)}



%\authorizationadd{本学位论文属于}
\ownersigncap{学位论文作者签名:~~~~~}
\supervisorsigncap{导师签名:~~~~~~~~~~~~~~~~~~~~~}
\signdatecap{签字日期:~~~~~~~~~年~~~~~~~~月~~~~~~~~日}




\cabstract{
	%这里是中文摘要
	摘要内容,用宋体小 4 号字,两端对齐,左缩进 2 个汉字符,行距为固定值 20 磅,段前空 0 磅,段后空 0 磅。论文摘要中不应出现图片、图表、表格或其他插图材料。论文关键词是为了文献标引工作从论文中选取出来用以表示全文主题内容信息的单词或术语,3-7 个,用宋体小 4 号字,每个关键词之间用分号间隔,两端对齐。

	摘要是论文内容的高度概括,应具有独立性和自含性,即不阅读论文的全文,就能获得必要的信息。摘要应包括本论文的目的、主要研究内容、研究方法、创造性成果及其理论与实际意义。
	摘要中不宜使用公式、化学结构式、图表和非公知公用的符号与术语,不标注引用文献编号,同时避免将摘要写成目录式的内容介绍。

	论文的摘要,是对论文研究内容的高度概括,其他人会根据摘要检索一篇学位论文,因此摘要应包括:对问题及研究目的的描述、对使用的方法和研究过程进行的简要介绍、对研究结论的简要概括等内容。摘要应具有独立性、自明性,应是一篇简短但意义完整的文章。

	通过阅读论文摘要,读者应该能够对论文的研究方法及结论有一个整体性的了解,因此摘要的写法应力求精确简明。论文摘要切忌写成全文的提纲,尤其要避免“第 1 章……;第 2 章……;……”这样的陈述方式。

}

%这里是中文关键词
\ckeywords{关键词 1;关键词 2; ……;关键词 5;关键词 6}



\eabstract{

	Externally pressurized gas bearing has been widely used in the field of aviation, semiconductor, weave, and measurement apparatus because of its advantage of high accuracy, little friction, low heat distortion, long life-span, and no pollution. In this thesis, based on the domestic and overseas researching……

}
%这里是英文关键词
\ekeywords{keyword 1; keyword 2; keyword 3; ……; keyword 6}

\makecover

\clearpage
\thispagestyle{empty}
