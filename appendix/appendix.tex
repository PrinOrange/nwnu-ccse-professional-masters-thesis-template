\addcontentsline{toc}{chapter}{附\quad 录} %添加到目录中
\chapter*{附\quad 录}
\setlength{\parindent}{2em}

\textcolor[rgb]{1.00,0.00,0.00}{(不需要的可不列此部分)}

附录是与论文内容密切相关、但编入正文又影响整篇论文编排的条理和逻辑性的一些资料,如某些重要的数据表格、计算程序、统计表等,是论文主体的补充内容,可根据需要设置。
附录的格式与正文相同,并依顺序用大写字母“A,B,C,……”或汉字“一,
二,三,……”编序号,如“附录 A,附录 B,附录 C,……”或“附录一,附录二,附录三,……”。只有一个附录时也要编序号,即附录 A。每个附录应有标题。附录序号与附录标题之间空一个字符,例如:“附录 A 甘肃省 2018 年度人口统计数据”。
附录中的图、表、数学表达式、参考文献等另行编序号,与正文分开,一律用阿拉伯数字编码,但在数码前冠以附录的序号,例如“图 A-1”,“表 B-2”,“式(C-3)”等。
